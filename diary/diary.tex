\documentclass{article}
\usepackage{braket}
\usepackage{amsmath}

\begin{document}
\section{January 19th 2023}
Working on understanding Majorana polarization. So we want to find two Hermitian Majoranas $\gamma^+$ and $\gamma^-$ (left/right Majorana) and form a non-local fermion
\begin{equation}
   f = \frac{1}{2}(\gamma^+ + i\gamma^-), 
\end{equation}
such that $f^\dag\ket{o} = \ket{e}$. Then we get $\ket{e} = \gamma^+\ket{o}$ and $\ket{e} = -i\gamma^-\ket{o}$. We can expand these Majoranas in the Hermitian site-Majoranas $\gamma_j^+ = d^\dag_j + d_j$ and $\gamma^-_j = i(d^\dag_j - d_j)$
\begin{equation}
\begin{split}
    \gamma^+ &= \sum_{j,s} a_j^s \gamma_j^s \\
    \gamma^- &= \sum_{j,s} b_j^s \gamma_j^s. 
\end{split}
\end{equation}
We then calculate e.g.,
\begin{equation}
    \begin{split}
        \braket{o|\gamma_j^+|e} &= \braket{o|\gamma_j^+ \gamma^+|o} = \braket{o|\gamma_j^+ \sum_{k,s} a_k^s \gamma_k^s|o} = \braket{o|a_j^+ + \sum_{(k, s)\neq(j,+)}a_k^s\gamma_j^+\gamma_k^s|o} \\
                                &= a_j^+ + \sum_{(k, s)\neq(j,+)} a_k^s \braket{o|\gamma_j^+\gamma_k^s|o}
    \end{split}
\end{equation}
but $\braket{o|\gamma_j^+\gamma_k^s|o}^*=\braket{o|\gamma_k^s\gamma_j^+|o} = -\braket{o|\gamma_j^+\gamma_k^s|o}$ (by using anti-commutation). Hence the second term is purely imaginary and we can write (analogously for the other coefficients)
\begin{equation}
    \begin{split}
        a_j^+ &= \operatorname{Re}\{\braket{o|\gamma_j^+|e}\} \\
        a_j^- &= \operatorname{Re}\{\braket{o|\gamma_j^-|e}\} \\
        b_j^+ &= -\operatorname{Im}\{\braket{o|\gamma_j^+|e}\} \\
        b_j^- &= -\operatorname{Im}\{\braket{o|\gamma_j^-|e}\} \\
    \end{split}
\end{equation}
since the $a,b$ coefficients are real ($\gamma^+$ and $\gamma^-$ are Hermitian). If the Hamiltonian is real (?), $a_j^- = b_j^+ = 0$ and we can skip taking the real part.

\begin{enumerate}
    \item What have we assumed when doing this? Have we assumed that we have good left/right majoranas and then derived their expansion?
\end{enumerate}

\section{January 20th 2023}
The Majorana polarization is defined in e.g., Aksenov 2020 as roughly (though not taking into account taking the real part as above)

\begin{equation}
    \text{MP} = \frac{\sum_j^{'} {a_j^+}^2 - {b_j^-}^2}{\sum_j^{'} {a_j^+}^2 + {b_j^-}^2}
\end{equation}
where the prime means to sum over half of the chain. This seems to depend on how much of $\gamma^-$ has leaked into the left side. 

\begin{enumerate}
    \item What is the meaning of the MP\@? Relation to original paper?
    \item Is it reasonably defined? What if we have an imaginary part? Spin?
    \item What if one Majoranas leaks to the opposite half but the other does not?
    \item Could you do a normal scalar product to calculate overlap?
    \item If no overlap, the denominator is not necessary since the sum of the coefficients squared is unity (see below).
\end{enumerate}
Proved that $\sum_k {a_k^+}^2 = 1$ by induction, anti-commutation relations and that the majoranas square to one, assuming that we only get the + coefficients. Also tried out the MP on the Poor mans geometry.
\section{January 23rd, 2023}
\begin{enumerate}
    \item Confused about Fourier transform of second quantization operators. Turns out it can be seen as a basis change between position and momentum space bases. See p. 16 to 18 in Flensberg Many body.
    \item Started trying to derive bulk energies of Kitaev chain with periodic boundary condition (Alicea).
    \item Discussed MP with Viktor, seems like the normalization is not good. One special case with overlap only on right side, still unity MP\@. To include spin one can subtract the spin part, such that it measures how lonely one of the spin species is. 
    \item Then I started looking through the derivation in Akhmerov to try and understand how the Kitaev chain emerges from local coupling.
\end{enumerate}

\section{January 24th, 2023}
Working on the derivation in Akhmerov, struggling with Maple. Might have a problem with one of my rules, giving the wrong sign in the new operators.
\section{Januray 25th, 2023}
Found problem with sign, just took other branch of $\Delta$. Struggling with getting the correct phases of the $a$ and $b$ operators in Maple, but I think I got it. Expressed the old $c$ operators in the new $a$ and $b$. Inserted it in Hamiltonian. The dot hamiltonian of course reduces to only $a^\dag a$ and $b^\dag b$ so only spin-orbit/rotation left. Found out what $e^{i\boldsymbol{\lambda}\boldsymbol{\sigma}}$ actually means, it is the matrix representation of the rotation operator $D(n,2\lambda)$ which rotates the spin $2\lambda$ around the $n$ axis. Also, very convenient to use Pauli matrix exponent formula. Very messy algebra however, try to get to Akhmerov. Don't understand how to/the reason why you can project onto the $a$ operators.

\section{Janurary 25th}
Looked at job openings before lunch. After lunch I managed to get to the expression for the tunneling in Akhmerov. Still confused about how you can remove all $b$ operators and what the conditions of doing it are. Thinking about the conditions on the parameters to get to the topological phase.
\end{document}
